\documentclass{article}
\usepackage{tikz,tikz-3dplot}
\usepackage{pgfplots}
\usepackage[margin=2cm,landscape]{geometry}

\usetikzlibrary{shapes}

\begin{document}
\makeatletter

\tikzoption{canvas is plane}[]{\@setOxy#1}
\def\@setOxy O(#1,#2,#3)x(#4,#5,#6)y(#7,#8,#9)%
  {\def\tikz@plane@origin{\pgfpointxyz{#1}{#2}{#3}}%
   \def\tikz@plane@x{\pgfpointxyz{#4}{#5}{#6}}%
   \def\tikz@plane@y{\pgfpointxyz{#7}{#8}{#9}}%
   \tikz@canvas@is@plane
  }

% enhance \tdplotsetmaincoords
\def\setMain#1#2{
    % disable \tikzset
    \let\tikzset\pgfutil@gobble
    % \tikzset is used here
    \tdplotsetmaincoords{#1}{#2}
    % repair \tikzset
    \def\tikzset{\pgfqkeys{/tikz}}
    \tikzset{Main/.style={x={(\raarot cm,\rbarot cm)},y={(\rabrot cm, \rbbrot cm)},z={(\racrot cm, \rbcrot cm)}}}
    % also setup Rotated coordinate
    \reRotate{0}{0}{0}
}

% enhance \tdplotsetrotatedcoords by appending the commented lines
% \resRotated means that the rotation is not accumulated.
\def\reRotate#1#2#3{
    % disable \tikzset
    \let\tikzset\pgfutil@gobble
    % \tikzset is used here
    \tdplotsetrotatedcoords{#1}{#2}{#3}
    % append the commented lines
    % see the source code tikz-3dplot.sty line 312-323
    \tdplotmult{\rcaeaa}{\rcarot}{\raaeul}
    \tdplotmult{\rcbeba}{\rcbrot}{\rbaeul}
    \tdplotmult{\rcceca}{\rccrot}{\rcaeul}
    \tdplotmult{\rcaeab}{\rcarot}{\rabeul}
    \tdplotmult{\rcbebb}{\rcbrot}{\rbbeul}
    \tdplotmult{\rccecb}{\rccrot}{\rcbeul}
    \tdplotmult{\rcaeac}{\rcarot}{\raceul}
    \tdplotmult{\rcbebc}{\rcbrot}{\rbceul}
    \tdplotmult{\rccecc}{\rccrot}{\rcceul}
    % see the source code tikz-3dplot.sty line 332-335
    \pgfmathsetmacro{\rcarc}{\rcaeaa+\rcbeba+\rcceca}
    \pgfmathsetmacro{\rcbrc}{\rcaeab+\rcbebb+\rccecb}
    \pgfmathsetmacro{\rccrc}{\rcaeac+\rcbebc+\rccecc}
    % repair \tikzset
    \def\tikzset{\pgfqkeys{/tikz}}
    \tikzset{Rotated/.style={x={(\raarc cm,\rbarc cm)},y={(\rabrc cm, \rbbrc cm)},z={(\racrc cm, \rbcrc cm)}}}%
}

% define a further-rotate version of \tdplotsetrotatedcoords
% \furtherRotate means that the rotation can be accumulated.
\def\furtherRotate#1#2#3{
    % before everything, deceive tikz-3dplot by letting main-coordinate to be rotated coordinate
    % in other words:
    %    \let\oldMain=\Main
    %    \let\Main=\Rotated
    %    \tdplotsetrotatedcoords{...}
    %    \let\Main=\oldMain
    \let\oldraarot\raarot\let\oldrabrot\rabrot\let\oldracrot\racrot
    \let\oldrbarot\rbarot\let\oldrbbrot\rbbrot\let\oldrbcrot\rbcrot
    \let\oldrcarot\rcarot\let\oldrcbrot\rcbrot\let\oldrccrot\rccrot
    \let\raarot\raarc    \let\rabrot\rabrc    \let\racrot\racrc
    \let\rbarot\rbarc    \let\rbbrot\rbbrc    \let\rbcrot\rbcrc
    \let\rcarot\rcarc    \let\rcbrot\rcbrc    \let\rccrot\rccrc
    %
    %
    % the following is like \tdplotsetrotatedcoords
    %
    %
    \reRotate{#1}{#2}{#3}
    %
    %
    % do not forget the \let\M=\oldM part
    %
    %
    \let\raarot\oldraarot\let\rabrot\oldrabrot\let\racrot\oldracrot
    \let\rbarot\oldrbarot\let\rbbrot\oldrbbrot\let\rbcrot\oldrbcrot
    \let\rcarot\oldrcarot\let\rcbrot\oldrcbrot\let\rccrot\oldrccrot
}

\def\Shift#1#2#3{
    \tikzset{Rotated}
    \pgfpointxyz{#1}{#2}{#3}
    \edef\temp@shift@vector{\noexpand\pgf@x\the\pgf@x\noexpand\pgf@y\the\pgf@y}
    \pgftransformshift{\temp@shift@vector}
}

\makeatother 

\begin{figure}
\centering
\tdplotsetmaincoords{0}{0}
\newcommand{\radius}{5}
\newcommand{\tuftRadius}{0.2}
\newcommand{\height}{1}
\newcommand{\tuftLength}{2}
\newcommand{\xTest}{2.5}
\newcommand{\zTest}{2.5}
\newcommand{\phiTest}{45}
\newcommand{\RTest}{3}
\newcommand{\thetaTest}{30}
\newcommand{\alphaTest}{45}
\newcommand{\betaTest}{30}



\begin{tikzpicture}[tdplot_main_coords]

    % \tdplotsetrotatedcoords{-150}{20}{150}
    \tdplotsetrotatedcoords{0}{0}{0}
    % ZYZ
    \reRotate{-90}{-90}{90}
    \furtherRotate{-90}{25}{50}
    
    \begin{scope}[tdplot_rotated_coords]
        \tikzstyle{every node}=[font=\small]
        \draw[thick,-latex] (0,0,0) -- (7,0,0) node[anchor=north east]{$x$};
        \draw[thick,-latex] (0,0,0) -- (0,7,0) node[anchor=north west]{$y$};
        \draw[thick,-latex] (0,0,0) -- (0,0,5) node[anchor=north east]{$z$};

        \tikzset{zxplane/.style={canvas is zx plane at y=#1,very thin}}
        \tikzset{yxplane/.style={canvas is yx plane at z=#1,very thin}}
        \begin{scope} [canvas is plane={O(0,0,0)x(1,0,0)y(0,1,0)}]
            \draw [thick](0,0) circle ({\radius});
        \end{scope}
        \begin{scope} [canvas is plane={O(0,0,{\height})x(1,0,{\height})y(0,1,{\height})}]
            \draw [thick](0,0) circle ({\radius});
        \end{scope}

        % cylinder side edges
        \draw [thick](0,{\radius},0) -- (0,{\radius},{\height});
        \draw [thick](0,-{\radius},0) -- (0,-{\radius},{\height});        
        \draw [thick]({\radius},0,0) -- ({\radius},0,{\height});
        \draw [thick](-{\radius},0,0) -- (-{\radius},0,{\height});

        % % R test circle
        % \begin{scope} [canvas is plane={O(0,0,{\height})x(1,0,{\height})y(0,1,{\height})}]
        %     \draw [thick,dotted](0,0) circle ({\RTest});
        % \end{scope}
        
        % z = Ztuft plane
        \filldraw[fill=orange, nearly transparent, dotted] (5,-5, {\height}) -- (5,5,{\height}) --  (-5,5,{\height}) -- (-5,-5,{\height}) -- (5,-5,{\height});
        \draw (5,5,{\height}) node[anchor=south west]{$z = Z_{tuft}$ plane};


        % Three position coordinates
        \draw[very thick, -latex, red] (0,0,{\height}) -- ({\RTest * cos(\thetaTest)},{\RTest * sin(\thetaTest)},{\height}) node[midway, below]{$R$};

        \tdplotdrawarc[very thick, -latex, green]{(0,0,{\height})}{\RTest}{0}{\thetaTest}{anchor=west}{$\theta$};

        \draw[very thick, -latex, blue] (0,0,0) -- (0,0,{\height}) node[midway,right]{$Z$};

        \draw [thick,dotted, -latex](0,0,{\height}) -- (5,0,{\height}) node[anchor= south east]{$x'$};


        
        % \node at (1.8,1,4)  { $P_1(r_1,\phi_1,z_1)$};
        % \draw[ultra thick,-latex](2,2.25,4) -- (3,3.45,4) node[anchor=north] {$\mathbf{a}_r$};
        % \draw[ultra thick,-latex](2,2.25,4) -- (1,2.5,4) node[anchor=north west] {$\mathbf{a}_\phi$};
        % \draw[ultra thick,-latex](2,2.25,4) -- (2,2.25,4.75) node[anchor=north west] {$\mathbf{a}_z$};
        % \draw [thick,->](4,0,0) arc (0:45:4 and 4.5);
        % \draw (3.6,2,0) node[anchor=north] {$\phi_1$};
        % \draw[ultra thick,-latex](0,0,0) -- (2,2.35,0);
        % \draw (1,1,0) node[anchor=north] {$r_1$};
        % \draw [ultra thick] (2,2.25,4)--(1.95,2.25,0);

        % \draw[ultra thick](0.1,0,4) -- (-0.1,0,4) node[anchor=south west] {$z_1$};

    \end{scope}       


    \Shift{\RTest * cos(\thetaTest)}{\RTest * sin(\thetaTest)}{\height}
    \furtherRotate{\thetaTest - 90}{0}{0}
    
    \begin{scope}[tdplot_rotated_coords]
        \begin{scope} [canvas is plane={O(0,0,0)x(1,0,0)y(0,-1,0)}]
            \fill [opacity=0.7,yellow,draw=black] (0,0) rectangle (-0.4,0.4);
        \end{scope}

        \draw [thick, dotted](-3,0,0) -- (3,0,0);        
        \draw [thick, dotted,  -latex](0,0,0) -- (0,0,{5-\height}) node[anchor= east]{$z'$};        
        \filldraw[fill=red, nearly transparent, dotted] (-3, 0 ,4.5) -- (3,0,4.5) --  (3,0,0) -- (-3, 0,0) -- (-3, 0,4.5);

        \draw [thick,  -latex] (-3,0,4.5) -- (-4,0,5);   

        \draw (-4,0,5.7) node{$\alpha$ angle rotation plane};
        \draw (-4,0,5.2) node{Passes tuft position and orthogonal to \textbf{R}};

        \begin{scope} [canvas is plane={O(0,0,0)x(1,0,0)y(0,0,1)}]
            \tdplotdrawarc[very thick, -latex, black]{(0,0,0)}{3}{90}{90 + \alphaTest }{anchor = south}{$\alpha_{tuft}$};
        \end{scope}

    \end{scope}

    % \furtherRotate{-90}{90}{90}
    % \tdplotdrawarc[very thick,tdplot_rotated_coords, -latex, black]{(0,0,0)}{3}{0}{\alphaTest }{anchor = east}{$\alpha_{tuft}$};
    % \furtherRotate{0}{-90}{0}
    
    \furtherRotate{0}{-\alphaTest}{0}

    \begin{scope}[tdplot_rotated_coords]
        \draw [thick, dotted](0,0,0) -- (0,0,5);        

        \filldraw[fill=blue, nearly transparent, dotted] (0, -3 ,5) -- (0,3,5) --  (0,3,0) -- (0, -3,0) -- (0, -3,5);

        \draw [thick,  -latex] (0,3,4.5) -- (0,3.5,4.5);   

        \draw (0,3.5,5) node[anchor = west]{$\beta$ angle rotation plane};
        \draw (0,3.5,4.5) node[anchor = west]{$\alpha = \alpha_{tuft}$};
        \draw (0,3.5,4.0) node[anchor = west]{Passes tuft position};


        \begin{scope} [canvas is plane={O(0,0,0)x(0,1,0)y(0,0,1)}]
            \tdplotdrawarc[very thick, -latex, black]{(0,0,0)}{3}{90}{90 - \betaTest }{anchor = south}{$\beta_{tuft}$};
        \end{scope}

    \end{scope}

    \furtherRotate{-90}{-\betaTest}{90}


    \begin{scope}[tdplot_rotated_coords]
        % \begin{scope} [canvas is plane={O({\xTest},{sin(\phiTest)},{cos(\phiTest)})x({1+\xTest},{sin(\phiTest)},{cos(\phiTest)})y({\xTest},{sin(\phiTest) + 1},{cos(\phiTest)})}]
        % \begin{scope} [canvas is plane={O(0,0,0)x(1,0,0)y(0,1,0)}]
        %     \draw [thick](0,0) circle ({\tuftRadius});
        % \end{scope}
        % \begin{scope} [canvas is plane={O(0,0,{\tuftLength})x(1,0,{\tuftLength})y(0,1,{\tuftLength})}]
        %     \draw [thick](0,0) circle ({\tuftRadius});
        % \end{scope}

        \newcommand{\n}{11}
        \newcommand{\h}{\tuftLength}
        \newcommand{\rl}{\tuftRadius}
        \newcommand{\rh}{\tuftRadius}
        % \path[draw,fill=white] plot[domain=0:2*pi,samples=4*\n] ({\rl*cos(\x r)}, {\rl*sin(\x r)}, 0);
        \foreach \t in {1,...,22} {
            \filldraw[fill=white] ({\rl*cos(\t*pi/\n r)},{\rl*sin(\t*pi/\n r)},0) -- ({\rl*cos((\t+1)*pi/\n r)},{\rl*sin((\t+1)*pi/\n r)},0) -- ({\rh*cos((\t+1)*pi/\n r)},{\rh*sin((\t+1)*pi/\n r)},\h) -- ({\rh*cos(\t*pi/\n r)},{\rh*sin(\t*pi/\n r)},\h) -- cycle;
        }
        % \path[draw,fill=white] plot[domain=0:2*pi,samples=4*\n] ({\rh*cos(\x r)}, {\rh*sin(\x r)}, \h);
        \draw [thick, -latex](0,0,0) -- (0,0,{\tuftLength + 2});

    \end{scope}






    % \begin{scope}[tdplot_rotated_coords]
    %     \draw [thick, dotted](0,0,0) -- (0,0,10);        
    % \end{scope}

    % \furtherRotate{90}{-90}{-90}
    % \tdplotdrawarc[very thick, tdplot_rotated_coords, -latex, black]{(0,0,0)}{3}{-270}{-270 - \betaTest}{anchor=south}{$\beta$};
    % \furtherRotate{-90}{-90}{90}




    % \furtherRotate{0}{\betaTest}{0}



\end{tikzpicture}
\end{figure}
\end{document}