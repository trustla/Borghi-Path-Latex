\documentclass{article}
\usepackage{tikz,tikz-3dplot}
\usepackage{pgfplots}

\usetikzlibrary{shapes}

\begin{document}
\makeatletter

\tikzoption{canvas is plane}[]{\@setOxy#1}
\def\@setOxy O(#1,#2,#3)x(#4,#5,#6)y(#7,#8,#9)%
  {\def\tikz@plane@origin{\pgfpointxyz{#1}{#2}{#3}}%
   \def\tikz@plane@x{\pgfpointxyz{#4}{#5}{#6}}%
   \def\tikz@plane@y{\pgfpointxyz{#7}{#8}{#9}}%
   \tikz@canvas@is@plane
  }

% enhance \tdplotsetmaincoords
\def\setMain#1#2{
    % disable \tikzset
    \let\tikzset\pgfutil@gobble
    % \tikzset is used here
    \tdplotsetmaincoords{#1}{#2}
    % repair \tikzset
    \def\tikzset{\pgfqkeys{/tikz}}
    \tikzset{Main/.style={x={(\raarot cm,\rbarot cm)},y={(\rabrot cm, \rbbrot cm)},z={(\racrot cm, \rbcrot cm)}}}
    % also setup Rotated coordinate
    \reRotate{0}{0}{0}
}

% enhance \tdplotsetrotatedcoords by appending the commented lines
% \resRotated means that the rotation is not accumulated.
\def\reRotate#1#2#3{
    % disable \tikzset
    \let\tikzset\pgfutil@gobble
    % \tikzset is used here
    \tdplotsetrotatedcoords{#1}{#2}{#3}
    % append the commented lines
    % see the source code tikz-3dplot.sty line 312-323
    \tdplotmult{\rcaeaa}{\rcarot}{\raaeul}
    \tdplotmult{\rcbeba}{\rcbrot}{\rbaeul}
    \tdplotmult{\rcceca}{\rccrot}{\rcaeul}
    \tdplotmult{\rcaeab}{\rcarot}{\rabeul}
    \tdplotmult{\rcbebb}{\rcbrot}{\rbbeul}
    \tdplotmult{\rccecb}{\rccrot}{\rcbeul}
    \tdplotmult{\rcaeac}{\rcarot}{\raceul}
    \tdplotmult{\rcbebc}{\rcbrot}{\rbceul}
    \tdplotmult{\rccecc}{\rccrot}{\rcceul}
    % see the source code tikz-3dplot.sty line 332-335
    \pgfmathsetmacro{\rcarc}{\rcaeaa+\rcbeba+\rcceca}
    \pgfmathsetmacro{\rcbrc}{\rcaeab+\rcbebb+\rccecb}
    \pgfmathsetmacro{\rccrc}{\rcaeac+\rcbebc+\rccecc}
    % repair \tikzset
    \def\tikzset{\pgfqkeys{/tikz}}
    \tikzset{Rotated/.style={x={(\raarc cm,\rbarc cm)},y={(\rabrc cm, \rbbrc cm)},z={(\racrc cm, \rbcrc cm)}}}%
}

% define a further-rotate version of \tdplotsetrotatedcoords
% \furtherRotate means that the rotation can be accumulated.
\def\furtherRotate#1#2#3{
    % before everything, deceive tikz-3dplot by letting main-coordinate to be rotated coordinate
    % in other words:
    %    \let\oldMain=\Main
    %    \let\Main=\Rotated
    %    \tdplotsetrotatedcoords{...}
    %    \let\Main=\oldMain
    \let\oldraarot\raarot\let\oldrabrot\rabrot\let\oldracrot\racrot
    \let\oldrbarot\rbarot\let\oldrbbrot\rbbrot\let\oldrbcrot\rbcrot
    \let\oldrcarot\rcarot\let\oldrcbrot\rcbrot\let\oldrccrot\rccrot
    \let\raarot\raarc    \let\rabrot\rabrc    \let\racrot\racrc
    \let\rbarot\rbarc    \let\rbbrot\rbbrc    \let\rbcrot\rbcrc
    \let\rcarot\rcarc    \let\rcbrot\rcbrc    \let\rccrot\rccrc
    %
    %
    % the following is like \tdplotsetrotatedcoords
    %
    %
    \reRotate{#1}{#2}{#3}
    %
    %
    % do not forget the \let\M=\oldM part
    %
    %
    \let\raarot\oldraarot\let\rabrot\oldrabrot\let\racrot\oldracrot
    \let\rbarot\oldrbarot\let\rbbrot\oldrbbrot\let\rbcrot\oldrbcrot
    \let\rcarot\oldrcarot\let\rcbrot\oldrcbrot\let\rccrot\oldrccrot
}

\def\Shift#1#2#3{
    \tikzset{Rotated}
    \pgfpointxyz{#1}{#2}{#3}
    \edef\temp@shift@vector{\noexpand\pgf@x\the\pgf@x\noexpand\pgf@y\the\pgf@y}
    \pgftransformshift{\temp@shift@vector}
}

\makeatother 

\begin{figure}
\centering
\tdplotsetmaincoords{0}{0}
\newcommand{\radius}{2}
\newcommand{\tuftRadius}{0.2}
\newcommand{\length}{5}
\newcommand{\tuftLength}{2}
\newcommand{\xTest}{3}
\newcommand{\yTest}{2.5}
\newcommand{\zTest}{2}


\newcommand{\phiTest}{45}
\newcommand{\alphaTest}{-45}
\newcommand{\betaTest}{-30}



\begin{tikzpicture}[tdplot_main_coords]

    % \tdplotsetrotatedcoords{-150}{20}{150}
    \tdplotsetrotatedcoords{0}{0}{0}
    \reRotate{-150}{20}{150}
    \begin{scope}[tdplot_rotated_coords]
        \tikzstyle{every node}=[font=\small]
        \draw[thick,-latex] (0,0,0) -- (7,0,0) node[anchor=north east]{$x$};
        \draw[thick,-latex] (0,0,0) -- (0,5,0) node[anchor=north west]{$y$};
        \draw[thick,-latex] (0,0,0) -- (0,0,10) node[anchor=north east]{$z$};
        \tikzset{zxplane/.style={canvas is zx plane at y=#1,very thin}}
        \tikzset{yxplane/.style={canvas is yx plane at z=#1,very thin}}

        
        \filldraw[fill=orange, nearly transparent, dotted] ({\xTest},-2, 7) -- ({\xTest},6,7) --  ({\xTest},6,-7) -- ({\xTest},-2,-7) -- ({\xTest},-2,7);
        \draw (5.5,6,0) node[anchor=south west]{$x = X_{tuft}$ plane};
        \draw (5.5,5.5,0) node[anchor=south west]{$\alpha$ angle rotation plane};

        \draw[very thick, -latex, red] (0,0,0) -- ({\xTest},0,0) node[midway, below]{$X_{tuft}$};

        \draw[very thick, -latex, green] ({\xTest},0,{\zTest}) -- ({\xTest},{\yTest},{\zTest}) node[midway, left]{$Y_{tuft}$};

        \draw[very thick, -latex, blue] ({\xTest},0,0) -- ({\xTest},0,{\zTest}) node[midway, below]{$Z_{tuft}$};

        \draw [thick,dotted]({\xTest},{\yTest},{\zTest-5}) -- ({\xTest},{\yTest},{\zTest+5}) ;       

    \end{scope}       

    \Shift{\xTest}{\yTest}{\zTest}

    \furtherRotate{0}{90}{0}
    \tdplotdrawarc[very thick,tdplot_rotated_coords, -latex, black]{(0,0,0)}{3}{180}{180 +\alphaTest }{anchor = east}{$\alpha_{tuft}$};
    \furtherRotate{0}{-90}{0}

    \furtherRotate{90}{-\alphaTest}{-90}

    \begin{scope}[tdplot_rotated_coords]
        \draw [thick, dotted](0,0,-5) -- (0,0,5);       
        
        \filldraw[fill=blue, nearly transparent, dotted] (0,0 ,4.5) -- (3,0,4.5) --  (3,0,-4.5) -- (0, 0,-4.5) -- (0, 0,4.5);

        \draw (3.5,0,0) node[anchor=south west]{$\alpha = \alpha_{tuft}$ plane, passing tuft position};
        \draw (3.5,-0.5,0) node[anchor=south west]{$\beta$ angle rotation plane};

    \end{scope}

    \furtherRotate{90}{-90}{-90}
    \tdplotdrawarc[very thick, tdplot_rotated_coords, -latex, black]{(0,0,0)}{3}{-270}{-270 + \betaTest}{anchor=north }{$\beta_{tuft}$};
    \furtherRotate{-90}{-90}{90}




    \furtherRotate{0}{-\betaTest}{0}


    \begin{scope}[tdplot_rotated_coords]

        \newcommand{\n}{11}
        \newcommand{\h}{\tuftLength}
        \newcommand{\rl}{\tuftRadius}
        \newcommand{\rh}{\tuftRadius}

        \foreach \t in {1,...,22} {
            \filldraw[fill=white] ({\rl*cos(\t*pi/\n r)},{\rl*sin(\t*pi/\n r)},0) -- ({\rl*cos((\t+1)*pi/\n r)},{\rl*sin((\t+1)*pi/\n r)},0) -- ({\rh*cos((\t+1)*pi/\n r)},{\rh*sin((\t+1)*pi/\n r)},\h) -- ({\rh*cos(\t*pi/\n r)},{\rh*sin(\t*pi/\n r)},\h) -- cycle;
        }

        \draw [thick, -latex](0,0,0) -- (0,0,{\tuftLength + 2});

    \end{scope}
\end{tikzpicture}
\end{figure}
\end{document}